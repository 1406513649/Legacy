\documentclass[a4paper]{article}
\usepackage[margin=1in]{geometry}
% generated by Docutils <http://docutils.sourceforge.net/>
\usepackage{fixltx2e} % LaTeX patches, \textsubscript
\usepackage{cmap} % fix search and cut-and-paste in Acrobat
\usepackage{ifthen}
\usepackage[T1]{fontenc}
\usepackage[utf8]{inputenc}
\setcounter{secnumdepth}{0}

%%% Custom LaTeX preamble
% PDF Standard Fonts
\usepackage{mathptmx} % Times
\usepackage[scaled=.90]{helvet}
\usepackage{courier}

%%% User specified packages and stylesheets

%%% Fallback definitions for Docutils-specific commands
% numeric or symbol footnotes with hyperlinks
\providecommand*{\DUfootnotemark}[3]{%
  \raisebox{1em}{\hypertarget{#1}{}}%
  \hyperlink{#2}{\textsuperscript{#3}}%
}
\providecommand{\DUfootnotetext}[4]{%
  \begingroup%
  \renewcommand{\thefootnote}{%
    \protect\raisebox{1em}{\protect\hypertarget{#1}{}}%
    \protect\hyperlink{#2}{#3}}%
  \footnotetext{#4}%
  \endgroup%
}

% hyperlinks:
\ifthenelse{\isundefined{\hypersetup}}{
  \usepackage[colorlinks=true,linkcolor=blue,urlcolor=blue]{hyperref}
  \urlstyle{same} % normal text font (alternatives: tt, rm, sf)
}{}


%%% Body
\begin{document}

To whom it may concern,

I am a longtime member of the Church of Jesus Christ of Latter Day Saints (the Church), a direct descendant of Hyrum Smith, born in the covenant, served in many leadership positions before serving a mission, during my mission, and after.  I am writing as a means to express the pain and depression I have felt in the Church and to chart a brighter path forward.


\section{My Story%
  \label{my-story}%
}

Like many, Mormonism has been a primary component of every aspect of my life.  At the age of 19, I dutifully entered in to service as a full time missionary.  My mission was a depressing time for me.  I never felt like I had a spiritual witness as promised in the scriptures and by church leaders.  I walked depressed, day in and day out, feeling like a fraud for testifying of the truthfulness of a church that I wasn't sure I believed in.  In an odd way, I didn't doubt the truthfulness of the Church, but only my ability to receive a spiritual witness.  I grew to believe that I was unworthy of receiving an answer from God and my unworthiness was the cause of His silence.  I did everything to show God that I was worthy.  I took seriously Boyd Packer's counsel that \textquotedbl{}a testimony is to be found in bearing it!\textquotedbl{}\DUfootnotemark{id1}{id12}{1} and bore testimony to thousands, maybe even tens of thousands of people; I wrote letters to every person I ever may have offended or transgressed from the time I was a young boy till the then present, fearing that my past sins still darkened my soul; I met with my mission president monthly and attended the temple weekly in search of answers;  I met with the mission's doctor regularly to treat my depression/anxiety; I read every book of scripture several times (I was the go to mission \textquotedbl{}scripture guru\textquotedbl{}); I fasted for days at a time; I followed Enos' example (Enos 1:4, BoM) and prayed day and night, I would literally go days without sleeping or eating in an effort to show God my willingness to sacrifice; I rarely wrote home to family because I feared that God would be unpleased if I disengaged from the work in any way.  In short, I completely \textquotedbl{}lost myself in the work\textquotedbl{}.

Funny story, at a fast and testimony meeting in one of my areas, my companion bore testimony of how much of a spiritual example I was to him because of my ceaseless prayers.  My depression grew deeper thinking that my prayers were answering \emph{his} prayers but not my own.

In the closing months of my mission I still had not received a spiritual witness.  So strong was the fear that leaving too soon was God's one last test for me, I requested that my mission be extended (knowing that doing so would cause me to miss my brother's wedding).  I ultimately came home at the original time, but only because of the pleadings of my mother (and a call from her to my mission president).

I came home a shell of my former self, physically and mentally.  I left a happy, enthusiastic, and athletic young man and returned dozens of pounds lighter and depressed because I feared that God had rejected my offering.  My physical appearance was truly frightening.  Out of embarrassment, I told people I had caught some third world virus, instead of admitting that my physical state was a reflection of my internal turmoil.  My tendency to fast and stay up praying for days, sometimes weeks, in a row certainly did not help.

To this day, I remember few details from my mission.  My wife and I occasionally talk to friends and acquaintances from my mission.  They go on and on about the fun times we had and the work we did - I recall none of them.  My wife is incredulous that I don't remember any of the details; it's a running joke that I am just a forgetful buffoon - except that I am an accomplished scientist and forget very little outside of my mission experience.  The jokes are a reminder of the pain I was feeling.  I've since learned that this pattern of forgetfulness is common to sufferers of PTSD.  I do not mean to conflate the plight of a privileged young man voluntarily serving his church with victims of abuse or other traumatic events, but I do see parallels in my suppressing much of the memory of my mission and \textquotedbl{}moving on\textquotedbl{}.

Fast forward many years, marriage, degrees in physics and mathematics, PhD in a field of applied physics, a few children, many more church callings across the country, and I was still in the same spiritual state as that confused and depressed young returned missionary.  I tried to take solace in a passage from a Doctrine and Covenants \textquotedbl{}Yea, behold, I will tell you in your mind and in your heart, by the Holy Ghost...\textquotedbl{} (D\&C 8:2).  I had to have known in my \emph{mind} as promised by the scripture, just not in my \emph{heart}.  I reasoned that of the many gifts of the spirit, I was given an abundance of reason and a shortage of feeling by the spirit (D\&C 46:11-13).  It was my duty to accept it and quit asking for what I already knew (D\&C 6:23).  But, I still longed for a personal spiritual affirmation that my life's work was not in vain.


\section{The Essays and Their (Unintended?) Consequences%
  \label{the-essays-and-their-unintended-consequences}%
}

The church recently released their gospel topics essays.  The essay on race and the priesthood stood out to me.  In particular, the following excerpt from one of the closing paragraphs\DUfootnotemark{id2}{id13}{2}
%
\begin{quote}

Today, the Church disavows the theories advanced in the past that black skin is a sign of divine disfavor or curse, or that it reflects unrighteous actions in a premortal life; mixed-race marriages are a sin; or that blacks or people of any other race or ethnicity are inferior in any way to anyone else.  Church leaders today unequivocally condemn all racism, past and present, in any form.

\end{quote}

I agree whole heartedly with the spirit of the statement and I am embarrassed of the times I tried to justify the Church's past racism as having come from God in some way.  The effect of this passage on my mind led to a paradigm shift in my approach to the Church and, in particular, its leaders.  It is well known that past Church leaders declared, without a shadow of a doubt, that black skin was a curse from God put upon its holder.  There has been debate as to whether the belief constitutes doctrine or just policy.  George A. Smith wrote to Prof. Lowry Nelson, an LDS sociologist\DUfootnotemark{id3}{id14}{3}
%
\begin{quote}

... Indeed, some of God's children were assigned to superior positions before the world was formed.  We are aware that some Higher Critics do not accept this, but the Church does.

Your (Prof. Lowry's) position seems to lose sight of the revelations of the Lord touching the preexistence of our spirits, the rebellion in heaven, and the doctrines that our birth into this life and the advantages under which we may be born, have a relationship in the life heretofore.

From the days of the Prophet Joseph even until now, it has been the \emph{doctrine of the Church, never questioned by any of the Church leaders, that the Negroes are not entitled to the full blessings of the Gospel}.

Your (Prof. Lowry's) ideas, as we understand them, appear to contemplate the intermarriage of the Negro and White races, a concept which has heretofore been most repugnant to most normal-minded people from the ancient patriarchs till now. God's rule for Israel, His Chosen People, has been endogamous. Modern Israel has been similarly directed. We are not unmindful of the fact that there is a growing tendency, particularly among some educators, as it manifests itself in this area, toward the breaking down of race barriers in the matter of \emph{intermarriage between white and blacks}, but it does not have the sanction of the Church and \emph{is contrary to Church doctrine}.

\end{quote}

(emphasis mine).  The Church's position with respect to interracial marriage, as espoused by Mr. Smith, seems to have originated from Brigham Young\DUfootnotemark{id4}{id20}{9}
%
\begin{quote}

Shall I tell you the law of God in regard to the African race? If the white man who belongs to the chosen seed mixes his blood with the seed of Cain, the penalty, under the law of God, is death on the spot. This will always be so

\end{quote}

The repulsiveness of Mr. Smith's and Mr. Young's statements should be self-evidentiary.  Apologists, and even the Church's essays, excuse away these, and other, statements by appealing to the notion that the Prophets are products of their time and sometimes ideas that seem repugnant today were acceptable then.  That notion is laughable.  The contemporaries of the Prophets were true visionaries like Harriet Tubman, Abraham Lincoln, Rosa Parks, Martin Luther King, Jr., etc.  These individuals fought to end racism while the Prophets fought to expand it to the afterlife.  The Prophets were not just products of their time, they were racists.

Whether these beliefs constituted doctrine or just policy is really inconsequential in its practice.  It is clear, however, that \emph{Church leaders believed it was doctrine} and taught it as such.  For us now to disavow and condemn these racist views is, in essence, condemning these very leaders.  This realization begs the following questions (predicated on the position that racism is not a fruit of the Spirit):
%
\begin{itemize}

\item If the Prophets, the men with whom God communicates directly, cannot discern their own racism, or the racism of their predecessors, from the Spirit, then how are we to discern our own feelings from the Spirit?

\item If the Prophets, the men with whom God communicates directly, cannot discern their own racism, or the racism of their predecessors, from the Spirit, then how are we to know that they are speaking by the spirit or just espousing their own prejudices?

\end{itemize}

Maybe more importantly
%
\begin{itemize}

\item What process controls have been put in place since 1978 to avoid a similar situation, a situation in which the Apostles and Prophets unanimously espoused teachings of Satan as coming from the Lord?

\end{itemize}

These questions are what led to my paradigm shift: \emph{There are no answers to these questions rooted in reality and a search for answers is an exercise in futility}.

Accepting the proposition that one can receive special answers from an extra terrestrial spiritual source calls in to question the entire scientific method - the most proven method by which knowledge is obtained.  As an illustration of the point, suppose, based on a spiritual confirmation of the truthfulness of the Book of Mormon, we conclude that the principal ancestors of native Americans are Hebrew (Introduction to the Book of Mormon, 1981).  Since DNA studies have conclusively shown that native Americans do not, in fact, descend from Hebrews (see the \textquotedbl{}Book of Mormon and DNA\textquotedbl{} gospel topics essay) we would necessarily conclude that the current state of DNA sequencing is unreliable (since it cannot be validated against \textquotedbl{}known\textquotedbl{} data).  The ramifications of this conclusion are far reaching.  Criminal justice, for instance, could not use results of DNA, doctors could not ethically use DNA to predict and treat genetic diseases, forensic scientists could not reliably use DNA to identify human remains, etc.

Clearly, the proposition that DNA sequencing is unreliable, based solely on a spiritual confirmation of the truthfulness of the Book of Mormon, is ridiculous.  This illustrates the that the notion that an extra terrestrial spiritual witness can lead one to truth is fundamentally flawed.  Because there is no independently verifiable method of distinguishing what we believe to be the Spirit from our own prejudices, we must conclude that the truth cannot be arrived at through feelings and/or emotions.

How then, do we acquire knowledge and confirm truths?  The scientific method.  In the scientific method, a hypothesis is formed and data collected to test the hypothesis.  If the collected data do not \emph{invalidate} the hypothesis, we tentatively accept it.  The degree of tentativeness is directly related to the quantity and quality of data used to test the hypothesis.  The scientific method is the only proven method at reliably arriving at verifiable conclusions.  It is safe to say that the scientific method has resulted in more blessings to the human race than any other event in human history.

I am embarrassed that it took so long for me to learn this truth.  Realizing that the Prophets are just men swayed by their own prejudices was eye opening to me and I thank the Church for writing the essays and helping me realize this.  I also realized that I am not a bad person, disfavored of God for not feeling of their (the Prophets) divine calling.  Indeed, the Prophets have provided no independently verifiable data that they are who they claim and my lifetime of depression and seeking spiritual answers was all for nought.


\section{Hiding the Truth%
  \label{hiding-the-truth}%
}

As I previously stated, that one phrase in the race and the priesthood essay caused a massive paradigm shift in me.  I have since read all of the Church's essays and many of the footnotes.  I can't help but think the Church has been involved in whitewashing its history from the days of Joseph Smith, Jr.  One should not be surprised by this assertion, Boyd Packer as a member of the Quorum of the Twelve Apostles once told institute teachers:
%
\begin{quote}

There is a temptation for the writer or the teacher of Church history to want to tell everything, whether it is worth or faith promoting or not.

Some things that are true are not very useful.

\end{quote}

Perhaps Mr. Packer was referring to these truths regarding Joseph Smith, Jr.:
%
\begin{itemize}

\item Mr. Smith had sexual relationships with teenagers behind his wife Emma's back;

\item Mr. Smith married Orson Hyde's wife while Mr. Hyde was serving a mission for the Church in Israel (what a shock that would be to come home from a mission and the Prophet took your wife while you were away!);

\item Mr. Smith married many women before he purported to have received the sealing power;

\item Mr. Smith was not sealed to Emma until...

\item Mr. Smith married the Partridge sisters (..., 18 and ..., 19), who had been left under his guardianship after their father died;

\item Mr. Smith lied about the practice of polygamy both in public and in private;

\item Mr. Smith's practice of polygamy contradicts the guidelines for its practice as described in D\&C 132;

\item Mr. Smith fraudulently set up a bank and profited from its failure at the expense of the saints;

\item Mr. Smith purported to translate records containing the history of a descendant of Pharaoh of Egypt known as the Kinderhook plates.  These plates were later discovered to be a hoax and a fraud.

\item Mr. Smith purported to translate papyri written by the hand of Abraham.  The writing are now contained in the Book of Abraham in the Pearl of Great Price.  The translation performed by Mr. Smith has since been shown to be incorrect and the Church, after decades of obfuscation admitted as such in the Historicity of the Book of Abraham Gospel Topics Essay.

\item Mr. Smith purported to translate the records of a band of Hebrews that came to the western hemisphere in/around 600 BC using a magic rock.  The writings are contained in what is known as the Book of Mormon.  No physical evidence for the Book of Mormon has been produced.  The Church recently released photos of Mr. Smith's magic rock and for the first time publicly described the translation process\DUfootnotemark{id5}{id22}{11}.

\item Mr. Smith used the same magic rock used to translate the Book of Mormon as a treasure seeking medium.  No treasure was ever reported to be found.

\item Mr. Smith gave several conflicting accounts of the first vision, the first coming at least twelve years after the purported event.

\end{itemize}


\section{The Bedrock of Our Faith%
  \label{the-bedrock-of-our-faith}%
}

Concerning Mr. Smith, President Joseph Fielding Smith said\DUfootnotemark{id6}{id15}{4}
%
\begin{quote}

Mormonism, as it is called, must stand or fall on the story of Joseph Smith.  He was either a prophet of God, divinely called, properly appointed and commissioned, or he was one of the biggest frauds this world has ever seen.  There is no middle ground.

\end{quote}

More recently, President Gordon B. Hinckley, confirmed this assertion, saying\DUfootnotemark{id7}{id16}{5}
%
\begin{quote}

... Our whole strength rests on the validity of that vision.  It either occurred or it did not occur.  If it did not, then this work is a fraud.  If it did, then it is the most important and wonderful work under the heavens.

\end{quote}

The story of Mr. Smith is one of lies and deception.  The story of Mr. Smith, as promulgated by the Church, is a demonstrable fraud.  Some may refer to it as a pious fraud (a fraud perpetuated with the motivation of increasing faith), but it is a fraud nonetheless.


\section{A Crossroads%
  \label{a-crossroads}%
}

At this point, I could still rationalize church participation, figuring that even though it was not \textquotedbl{}true\textquotedbl{}, in the literal sense, it was still a good place to learn morals and values.  That assumption, however, was based on the false notion that religious people are more \textquotedbl{}moral\textquotedbl{} than non-religious.  Data show otherwise, demonstrating that people are good/moral or bad/immoral independent of religious affiliation\DUfootnotemark{id8}{id17}{6}.  The church validated this truth by providing an example of a religion, purporting to be God's true religion, that spreads bigotry and hate when in November 2015 the policy regarding LGBT peoples was publicly exposed.  Knowing that homosexuality is a biological characteristic of a person, as much as skin or hair color \hyperlink{ref}{[ref]}, how can we be so bigoted?  Knowing now that the Prophets have been so wrong on matters of race, how can we know the source of the policy change comes from some higher spiritual plain?  Clearly, the answer is \emph{we can't}.  What we do know, however, is that the policy change is not based on science/reason/data and is inhumane.  The policy leads to an increase in bigotry and hate and to a decrease in love and community.

I am now at a painful crossroads.  Can I, in good conscience, be associated with an organization that is led, in part, by ethnocentric bigotry and uses its members and political clout to enact policies that are, at the very least, not Christ-like?  If I continue my affiliation, will my children encounter the same depression and difficulties that I have had?  Can I allow my children to be affiliated with this organization, knowing I will have to deprogram much of what they learn from it?  Could I continue my affiliation, with my views known, and be accepted as a member of the community?  Will my children and/or wife be ostracized by other members because their father is an \textquotedbl{}apostate\textquotedbl{}?


\section{Taffy Pulling%
  \label{taffy-pulling}%
}

Recently Jeffrey Holland (cc.'d) said\DUfootnotemark{id10}{id18}{7}
%
\begin{quote}

Don't you dare bail!  I'm so furious with people who leave this church.  I don't know whether furious is a good apostolic word (crowd laughs).  But I am.  And I say, what on earth kind of conviction is that?  What kind of paddy-cake, taffy-pulled experience is that?  As if none of this mattered, as if nothing in our contemporary life mattered?  As if this is all supposed to be exactly the way I want it and answered every one of my questions and pursue this and occupy that, decide this, and then maybe I'll be a Latter-day Saint.  Well, there is too much Irish in me for that.

... Does it ever dawn on anybody that God might be tired?  Or that Christ might be tired?  He's people tired.  He's blessing tired.  He's parable tired.  He's sermon tired.  Everywhere he goes he's tired!  It's people people people.  Problems problems problems.  ... Bless my father, heal my wounds. He's exhausted.

\end{quote}

\emph{Fuck you} Mr. Holland.  Far from a \textquotedbl{}paddy-cake, taffy-pulled\textquotedbl{} experience, I have given much of my life to the Church (and I am not \emph{paid} to do so).  I have given years of my life, large sums of money, and my mental health for this organization.  How dare \emph{you} minimize the pain I am feeling because of the lies that \emph{you} and the Church have perpetuated.  In some ways I hate the place I now find myself.  I did not ask for this, many of the same feelings of depression from my mission have resurfaced.  Discovering the Church's lies has caused undo strain in my marriage, and left me in a difficult situation with my family.  It has been a long and painful process.

Contrast Mr. Holland's statements with those of Deiter Uchtdorf\DUfootnotemark{id11}{id19}{8}
%
\begin{quote}

The search for truth has led millions of people to The Church of Jesus Christ of Latter Day Saints.  However, there are some who leave the Church they once loved.  One might ask, \textquotedbl{}If the gospel is so wonderful, why would anyone leave?\textquotedbl{}  Sometimes we assume it is because they have been offended or lazy or sinful.  Actually, it is not that simple...

Some of our dear members struggle for years with the question whether they should separate themselves from the Church.

\end{quote}

I do not agree with Mr. Uchtdorf's equivocation of the gospel to the Church, but agree with his sentiment.


\section{Where Now?%
  \label{where-now}%
}

If we were honest, we would all admit to the high level of uncertainty that exists about what is to come after this life.  If we would admit that, perhaps we would stop banking on promises of IOUs in the hereafter and focus on maximizing our time now.  Rather than diminishing our value for life, understanding the finite nature of our existence should enlarge our capacity for hope and love.  I have a new appreciation for my family and wonderful wife.  I hope for nothing more than an eternity with them, but in the meantime, I want to maximize our happiness during this time we have together.  I've discovered a renewed desire to serve and help others, to maximize their happiness, and/or minimize their pain.  I would love if the Church were to play a role in that vision.  Unfortunately, Mr. Uchtdorf's loving view seems to be the minority view in the Church.  Mr. Hollands view of \textquotedbl{}us against them\textquotedbl{} and exclusion of those who are different seem to be the norm.  Sadly, I am starting to see the Church having a diminished role in my life moving forward.  More and more the Church seems to be an echo chamber of devotion to the current Prophet and his views, instead of a place where people can gather, learn, love, and grow with one another.
%
\DUfootnotetext{id12}{id1}{1}{%
Boyd K. Packer, \textquotedbl{}The Candle of the Lord\textquotedbl{}, Ensign (Jan. 1983)
}
%
\DUfootnotetext{id13}{id2}{2}{%
Race and priesthood essay
}
%
\DUfootnotetext{id14}{id3}{3}{%
Lowry letters
}
%
\DUfootnotetext{id15}{id6}{4}{%
J.F. Smith, Doctrines of Salvation, vol 1, p. 188-189
}
%
\DUfootnotetext{id16}{id7}{5}{%
G.B. Hinckley, \textquotedbl{}The Marvelous Foundation of Our Faith\textquotedbl{}, General Conference Address, October 2002
}
%
\DUfootnotetext{id17}{id8}{6}{%
it
}
%
\DUfootnotetext{id18}{id10}{7}{%
J.R. Holland, Tempe Arizona Devotional, April 2016
}
%
\DUfootnotetext{id19}{id11}{8}{\newcounter{listcnt0}
\begin{list}{\Alph{listcnt0}.}
{
\usecounter{listcnt0}
\addtocounter{listcnt0}{3}
\setlength{\rightmargin}{\leftmargin}
}

\item Uchtdorf, \textquotedbl{}Come Join with Us\textquotedbl{}, General Conference Address, October 2013
\end{list}
}
%
\DUfootnotetext{id20}{id4}{9}{\setcounter{listcnt0}{0}
\begin{list}{\Alph{listcnt0}.}
{
\usecounter{listcnt0}
\addtocounter{listcnt0}{1}
\setlength{\rightmargin}{\leftmargin}
}

\item Young, Journal of Discourses, 10:110, March, 1863.
\end{list}
}
%
\DUfootnotetext{id21}{id21}{10}{%
J.L. Lund, The Church and the Negro, 1967, p. 54.
}
%
\DUfootnotetext{id22}{id5}{11}{%
Ensign, October 2015
}
\begin{figure}[b]\raisebox{1em}{\hypertarget{ref}{}}[ref]
Here
\end{figure}

\end{document}
